\documentclass{article}

\title{COMPSCI2026 Algorithmics \\ \vspace{4mm} 
 Assessed Exercise -- Status and Implementation Reports}

\author{\bf Insert your name\\ \bf and matriculation number here}

\date{\today}

\begin{document} 
\maketitle

\section*{Status report}

The program for both wordladder and Dijkstra's algorithm compile and appear to run correctly uin the test cases provided. 

Both programs were created without the use of AI.
\section*{Implementation report}

\begin{itemize}
\item[(a)] 
Each word in the dictionary is a vertex, an edge exists between two vertices if their words differ by exactly one letter. The graph is represented using adjacency lists for storage and traversal.#
\begin{itemize}
    \item \textbf{Graph Initialisation:} 
    The dictionary file is read and each word is added as a vertex in the graph. Each word is compared with every other word in the dictionary and if two words differ by exactly one letter, an edge is added between them in the form of adding them to each others in the adjacency list. 

    \item \textbf{Shortest Path Search:} 
    A Breadth-First Search algorithm is used to determine the shortest word ladder. BFS was used as it is ideal for unweighted graphs as it guarantees the shortest path in terms of the number of edges. The algorithm starts from the vertex belonging to the start word and marks it as visited, then adds it to a queue. It continues traversing until the end word is reached or all possibilities are used up.

    \item \textbf{Path Construction:} 
    Each vertex stores its predecessor during the BFS. The path is reconstructed by backtracking from the end word to the start word using the predecessors. If a path is not possible, then the program reports that no ladder is possible.

    \item \textbf{Efficiency Improvements:} 
    \begin{itemize}
        \item A HashMap is used to map words to their indices in the graph, allowing for a  O(1) lookup for vertex indices.
        \item The adjacency list representation minimizes memory usage compared to an adjacency matrix because it only stores edges that exist.
        \item BFS uses linear time complexity in relation  to the number of vertices and edges.
    \end{itemize}
\end{itemize}

\item[(b)]
Here, explain how you implemented the Dijkstra’s algorithm for finding shortest paths. Include a discussion of any steps that you took to improve efficiency.
\end{itemize}

\section*{Empirical results}

Summarise the running times of your two programs on the data sets provided. If the program fails to terminate in, say, two minutes, simply report non-termination. 

\end{document}
